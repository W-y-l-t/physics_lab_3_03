\documentclass[12pt]{article}
\usepackage{preamble}
\usepackage{longtable}

\geometry{a4paper, left=2.5cm, right=2.5cm, top=2cm, bottom=2.5cm}

\pagestyle{plain}

\newcommand{\placeholder}[1]{{\color{magenta}#1}}

\chead{
    \begin{minipage}{1\linewidth}
        \begin{wrapfigure}{r}{0pt}
            \includegraphics[height=1cm]{images/logo}
        \end{wrapfigure}
        {
            \centering
            \sffamily\scriptsize
            \textbf{
                Санкт-Петербургский национальный исследовательский университет \\
                информационных технологий, механики и оптики}
            %th3p4g
            \vspace{2mm}

            \quad\quad\quad\quad\quad\quad\ \textbf{УЧЕБНЫЙ ЦЕНТР ОБЩЕЙ ФИЗИКИ ФТФ}
        }
    \end{minipage}
}



\begin{document}
    \vspace*{2\baselineskip}

    \thispagestyle{fancy}

    \noindent
    \textbf{Группа} \underline{M3200\hspace{4.5cm}} \hfill \textbf{К работе допущен} \underline{\hspace{4cm}} \\[0.5cm]
    \textbf{Студенты} \underline{Комашко А., Семёнов Д.\hspace{0.45cm}} \hfill \textbf{Работа выполнена} \underline{\hspace{4cm}} \\[0.5cm]
    \textbf{Преподаватель} \underline{Шоев В. И.\hspace{1.9cm}} \hfill \textbf{Отчет принят} \underline{\hspace{4.85cm}} \\



    \begin{center}
    {\huge \textbf{Рабочий протокол и отчёт по\\ лабораторной работе № 3.03}}

        \smallvspace

        {\Large Определение удельного заряда электрона}
    \end{center}


    \noindent
    1. \textbf{Цель работы.}

    \begin{enumerate}
        \item \placeholder{Цель №1}

        \item \placeholder{Цель №N}
    \end{enumerate}

    \mediumvspace

    \noindent
    2. \textbf{Задачи, решаемые при выполнении работы.}

    \begin{enumerate} 
        \item Провести измерения зависимости анодного тока $I_a$ вакуумного
        диода от величины тока в соленоиде при различных значениях
        анодного напряжения.
        \item Найти значение коэффициента связи между током соленоида
        и магнитным полем $B$ внутри него.
        \item Построить графики зависимостей $I_a$ от $B$ и определить по
        ним величины критических полей для каждого значения анодного
        напряжения.
        \item По значениям критического поля найти величину удельного
        заряда электрона и оценить ее погрешность.
    \end{enumerate}

    \mediumvspace

    \noindent
    3. \textbf{Объект исследования: } 
    
    Поток электронов внутри диода под воздействием магнитного поля соленоида

    \mediumvspace

    \noindent
    4. \textbf{Метод экспериментального исследования.}

    \placeholder{Метод исследования}

    \mediumvspace

    \noindent
    5. \textbf{Рабочие формулы и исходные данные.}

    \text{Сила Лоренца}
\[
    \vec{F}_L = \vec{F}_E + \vec{F}_M = -e\vec{E} -e [\vec{v}, \vec{B}]
\]

\text{Кинетическая энергия электрона в точке c потенциалом $\varphi$}
\[
    e\varphi = \frac{mv^2}{2} = \frac{m (v_r^2 + v_\varphi^2)}{2}
\]

\text{При критическом значении магнитной индукции $B_c$}
\[
    eU = \frac{m v_\varphi^2}{2}
\]

\text{Азимутальная скорость электрона}
\[
    v_\varphi = \frac{e B r}{2m}
\]

\text{Удельный заряда электрона}
\[
    \frac{e}{m} = \frac{8 U}{B_c^2 r_a^2}
\]

\text{Магнитная индукция внутри соленоида}
\[
    B_c = \mu_0 N \frac{I_c}{\sqrt{l^2 + d^2}}
\]

\text{Где:}
\begin{center}
    $\mu_0 = 4\pi\cdot10^{-7} \frac{\text{Гн}}{\text{м}} - \text{магнитаная постоянная}$ \\
    $N - \text{число витков соленоида}$ \\
    $l - \text{длина соленоида}$ \\
    $d - \text{диаметр соленоида}$
\end{center}

\text{Итоговая формула для расчета удельного заряда электрона}
\[
    \frac{e}{m} = \frac{8U (l^2 + d^2)}{(\mu_0 r_a N I_c)^2}
\]

\textbf{Исходные данные}
\begin{center}
    $r_a = 0.003$ м — радиус анода. \\
    $d = 37$ мм — диаметр соленоида. \\
    $l = 36$ мм — длина соленоида. \\
    $N = 1500$ — число витков. \\
\end{center}

    \mediumvspace

    \noindent
    6. \textbf{Измерительные приборы.}

    \smallvspace

    \begin{center}
    \begin{tabular}{|c|m{5cm}|C{2cm}|C{2cm}|C{2cm}|c|}
        \hline
        № п/п & Наименование            & Предел\newlineизмерений & Цена\newlineделения     & $\Delta_n$ \\
        \hline
        1     & Мультиметр MY64 & 0 - 10  A     & 1  мA  & 1  мА     \\
        \hline
        2     & Мультиметр MY65 & 0 - 2  мА     & 1 мкА  & 1 мкА    \\
        \hline

    \end{tabular}

    \smallvspace

    \textit{Таблица 1.} Измерительные приборы
\end{center}

    \mediumvspace

    \noindent
    7. \textbf{Схема установки. (перечень схем, которые составляют Приложение 1).}

    \hyperlink{schema1}{Схема установки} прилагается в Приложении 1

    \mediumvspace

    \noindent
    8. \textbf{Результаты прямых измерений и их обработки (таблицы, примеры расчетов).}

    \input{08_measure_results}

    \mediumvspace

    \noindent
    9. \textbf{Расчет результатов косвенных измерений (таблицы, примеры расчетов).}

    \begin{itemize}
    \item Заметим, что левая часть графика имеет теоретическую линейную зависимость. В силу 
    явлений, имеющих сложную природу описания, правую часть вычислений будем аппроксимировать 
    полиномом 3-ей степени.

    \item Найдем линейную зависимость для первых 8 полученных значения с помощью метода наименьших квадратов. 
    Пример для $U = 9$ В:

    $\overline{I_{L1-8}} = 82.5, \qquad \overline{I_{a1-8}} = 0.218$

    $\displaystyle b = \frac{\sum_{i = 1}^8 (I_{Li} - \overline{I_{L1-8}})(I_{ai} - \overline{I_{a1-8}})}{\sum_{i = 1}^8 (I_{Li} - \overline{I_{L1-8}})^2} = -1.19 \cdot 10^{-6}$

    $a = \overline{I_{a1-8}} - b \cdot \overline{I_{L1-8}} = 0.218$

    Получаем данные линейные аппроксимации:

    \begin{center}
        \begin{tabular}{|c|c|}
            \hline
            $U$, В & Линейная аппроксимация ($I_L$ и $I_a$ в мА) \\
            \hline
            $9$ & $I_a \approx -1.19 \cdot 10^{-6} \cdot I_L + 0.218$ \\
            \hline
            $10.5$ & $I_a \approx -5.89 \cdot 10^{-6} \cdot I_L + 0.264$ \\
            \hline
            $13.5$ & $I_a \approx -1.66 \cdot 10^{-5} \cdot I_L + 0.357$ \\
            \hline
        \end{tabular}

        \smallvspace

        \textit{Таблица 2.} Аппроксимации $I_a = I_a(I_L)$ при $B < B_\text{кр}$

    \end{center}

    \item Найдем полиномы 3-ей степени для аппроксимации зависимости для $B > B_\text{кр}$. 
    Решая данную систему уравнений для $a, b, c, d$

    \[\begin{cases} \displaystyle 
        a \sum_{i = 9}^{20} I_{Li}^3 + b \sum_{i = 9}^{20} I_{Li}^2 + c \sum_{i = 9}^{20} I_{Li} + d n = \sum_{i = 9}^{20} I_{Li}   \\
        a \sum_{i = 9}^{20} I_{Li}^4 + b \sum_{i = 9}^{20} I_{Li}^3 + c \sum_{i = 9}^{20} I_{Li}^2 + d \sum_{i = 9}^{20} I_{Li}^1 = \sum_{i = 9}^{20} I_{ai} I_{Li}   \\
        a \sum_{i = 9}^{20} I_{Li}^5 + b \sum_{i = 9}^{20} I_{Li}^4 + c \sum_{i = 9}^{20} I_{Li}^3 + d \sum_{i = 9}^{20} I_{Li}^2 = \sum_{i = 9}^{20} I_{ai}^2 I_{Li}   \\
        a \sum_{i = 9}^{20} I_{Li}^6 + b \sum_{i = 9}^{20} I_{Li}^5 + c \sum_{i = 9}^{20} I_{Li}^4 + d \sum_{i = 9}^{20} I_{Li}^3 = \sum_{i = 9}^{20} I_{ai}^3 I_{Li}  \\
    \end{cases}\]

    получаем такие полиномы 3-ей степени:
    
    \begin{center}
        \begin{tabular}{|c|c|}
            \hline
            $U$, В & Кубическая аппроксимация ($I_L$ и $I_a$ в мА) \\
            \hline
            $9$ & $I_a \approx -5.89 \cdot 10^{-9} \cdot I_L^3 + 8.91 \cdot 10^{-6} \cdot I_L^2 + 
                -4.51 \cdot 10^{-3} \cdot I_L + 0.802$ \\
            \hline
            $10.5$ & $I_a \approx -1.72 \cdot 10^{-8} \cdot I_L^3 + 2.14 \cdot 10^{-5} \cdot I_L^2 + 
                -9.16 \cdot 10^{-3} \cdot I_L + 1.397$ \\
            \hline
            $13.5$ & $I_a \approx -3.79 \cdot 10^{-8} \cdot I_L^3 + 4.54 \cdot 10^{-5} \cdot I_L^2 + 
                -1.84 \cdot 10^{-6} \cdot I_L + 2.638$ \\
            \hline
        \end{tabular}

        \smallvspace

        \textit{Таблица 3.} Аппроксимации $I_a = I_a(I_L)$ при $B > B_\text{кр}$

    \end{center}

    \item Пересечения данных функций дают такие точки: 
    
    \begin{itemize}
        \item для $U = 9$ В $\ I_{L_\text{кр}} = 194.31$ мА
        \item для $U = 10.5$ В $\ I_{L_\text{кр}} = 208.09$ мА
        \item для $U = 13.5$ В $\ I_{L_\text{кр}} = 223.84$ мА
    \end{itemize}

    \item Для полученных значений критической силы тока вычисляем критическое значение магнитного поля
    и удельный заряд электрона. Пример для $U = 9$ В:

    $\displaystyle B_\text{кр} = \mu_0 I_{L_\text{кр}} N \frac{1}{\sqrt{l^2 + d^2}} = 7.094 \cdot 10^{-3}$ Тл

    $\displaystyle \frac{e}{m} = \frac{8 U}{B^2_\text{кр} r^2_a} = 1.589 \cdot 10^{11}$ Кл/кг

    Результаты прилагаются в \hyperlink{table2}{таблице 5} в Приложении 2

    \item При помощи метода наименьших квадратов найдем угловой коэффициент прямой, аппроксимирующей 
    зависимость $B^2_\text{кр}$ от $U$:

    $\overline{U} = 11, \qquad \overline{B^2_\text{кр}} = 5.829 \cdot 10^{-5}$

    $\displaystyle b = \frac{\sum_{i = 1}^3 (U_i - \overline{U})(B^2_{i\text{кр}} - \overline{B^2_\text{кр}})}{\sum_{i = 1}^3 (U_i - \overline{U})^2} = 3.568 \cdot 10^{-6}$

    $a = \overline{B^2_\text{кр}} - b \cdot \overline{U} = 1.903 \cdot 10^{-5}$

    \item Подставляя в формулу значение $\displaystyle \frac{\Delta B^2_\text{кр}}{\Delta U} = 3.568 \cdot 10^{-6}$ получаем $\displaystyle \frac{e}{m} = 2.490 \cdot 10^{11} $ Кл/кг

\end{itemize}


    \mediumvspace

    \noindent
    10. \textbf{Расчёт погрешности измерений.}
    
    \begin{enumerate}
    \item Найдем погрешность для $\frac{e}{m}$, рассматрия их как результаты многократных измерений
    
    $\displaystyle \overline{\left(\frac{e}{m}\right)} = 1.66 \cdot 10^{11}$

    $\displaystyle S_{\frac{e}{m}} = \sqrt{\frac{\sum_{i = 1}^n \left(\left(\frac{e}{m}\right)_i - \overline{\left(\frac{e}{m}\right)}\right)^2}{3 \cdot 2}} = 6.49 \cdot 10^9$

    $\displaystyle \Delta \frac{e}{m} = t_{\alpha,3} S_{\frac{e}{m}} = 2.791 \cdot 10^{10}$, где $t_{\alpha,3} = 4.3$ - коэффициент Стьюдента для $\alpha = 0.95$

    $\displaystyle \varepsilon_{\frac{e}{m}} = \frac{\Delta \frac{e}{m}}{\overline{\left(\frac{e}{m}\right)}} = 16.7\%$
\end{enumerate}

    \mediumvspace

    \noindent
    11. \textbf{Графики (перечень графиков, которые составляют Приложение 3).}

    \begin{itemize}
        \item \hyperlink{diagram1}{График 1} Зависимость $I_a = I_a(I_L)$
        \item \hyperlink{diagram2}{График 2} Зависимость $I_a/I_L$ от $I_L$
        \item \hyperlink{diagram3}{График 3} Зависимость $B^2_\text{кр}$ от $U$
    \end{itemize}

    \mediumvspace

    \noindent
    12. \textbf{Окончательные результаты.}

    $\frac{e}{m} = (1.66 \pm 0.28) \cdot 10^{11} $ Кл/кг $\qquad \varepsilon_{\frac{e}{m}} = 16.7\% \qquad \alpha = 0.95$

    \mediumvspace

    \noindent
    13. \textbf{Выводы и анализ результатов работы.}

    Сравнивая с табличным значением удельного заряда электрона, равным $1.76 \cdot 10^{11}$ Кл/кг, можем заключить,
что полученные нами результаты имеют довольно приближенную оценку к реальному значению. Однако значение, 
полученное при помощи углового коэффициента $\frac{\Delta B^2_\text{кр}}{\Delta U}$, имеет оценку хуже.

Среди наиболее вероятных источников погрешности можно назвать метод нахождения критического значения
величины магнитного поля путем аппроксимации зависимости.


    \clearpage

    \begin{center}
        \LARGE
        \textbf{Приложение 1. Схема установки}
    \end{center}

    \mediumvspace

    \hypertarget{schema1}{}

\begin{center}
    \includegraphics[width=15cm]{images/scheme1}

    \smallvspace

    \textit{Рисунок 1.} Принципиальная электрическая схема измерительного стенда
    (цепь питания накала катода не показана)
\end{center}






    \clearpage

    \begin{center}
        \LARGE
        \textbf{Приложение 2. Таблицы измерений и расчётов}
    \end{center}

    \mediumvspace

    \begin{center}
    \hypertarget{table2}{}

    \renewcommand{\arraystretch}{1.8}

    \begin{longtable}{|C{1.6cm}|C{2cm}|C{2cm}|C{2cm}|C{2cm}|C{2cm}|C{2cm}|}
        \hline
        \multirow{3}{*}{№ опыта} & \multicolumn{6}{c|}{Анодное напряжение} \\
        \cline{2-7}
        & \multicolumn{2}{c|}{$U = 9$ В} & \multicolumn{2}{c|}{$U = 10.5$ В} & \multicolumn{2}{c|}{$U = 13.5$ В} \\
        \cline{2-7}
        & $I_L$, мА & $I_a$, мА & $I_L$, мА & $I_a$, мА & $I_L$, мА & $I_a$, мА \\
        \hline
        1 & $0$ & $0.218$ & $0$ & $0.264$ & $0$ & $0.357$\\
        \hline
        2 & $23$ & $0.218$ & $29$ & $0.264$ & $24$ & $0.357$\\
        \hline
        3 & $42$ & $0.218$ & $59$ & $0.264$ & $52$ & $0.357$\\
        \hline
        4 & $67$ & $0.218$ & $80$ & $0.264$ & $77$ & $0.358$\\
        \hline
        5 & $95$ & $0.219$ & $102$ & $0.264$ & $97$ & $0.358$\\
        \hline
        6 & $123$ & $0.219$ & $131$ & $0.265$ & $118$ & $0.358$\\
        \hline
        7 & $145$ & $0.218$ & $155$ & $0.266$ & $148$ & $0.359$\\
        \hline
        8 & $165$ & $0.217$ & $184$ & $0.264$ & $176$ & $0.360$\\
        \hline
        9 & $192$ & $0.212$ & $209$ & $0.257$ & $202$ & $0.355$\\
        \hline
        10 & $213$ & $0.203$ & $230$ & $0.228$ & $231$ & $0.339$\\
        \hline
        11 & $239$ & $0.153$ & $259$ & $0.162$ & $256$ & $0.254$\\
        \hline
        12 & $260$ & $0.126$ & $284$ & $0.125$ & $279$ & $0.198$\\
        \hline
        13 & $283$ & $0.098$ & $314$ & $0.103$ & $299$ & $0.167$\\
        \hline
        14 & $310$ & $0.081$ & $342$ & $0.088$ & $322$ & $0.148$\\
        \hline
        15 & $332$ & $0.070$ & $362$ & $0.079$ & $346$ & $0.134$\\
        \hline
        16 & $362$ & $0.058$ & $383$ & $0.070$ & $374$ & $0.114$\\
        \hline
        17 & $382$ & $0.052$ & $403$ & $0.063$ & $400$ & $0.100$\\
        \hline
        18 & $405$ & $0.046$ & $424$ & $0.056$ & $424$ & $0.091$\\
        \hline
        19 & $427$ & $0.041$ & $445$ & $0.051$ & $447$ & $0.082$\\
        \hline
        20 & $449$ & $0.037$ & $467$ & $0.047$ & $472$ & $0.073$\\
        \hline
    \end{longtable}

    \smallvspace

    \textit{Таблица 4.} Зависимость напряжения $U_R$ от тока в соленоиде
\end{center}

    \begin{center}
    \hypertarget{table3}{}

    \renewcommand{\arraystretch}{1.8}

    \begin{tabular}{|c|C{2.5cm}|C{2.5cm}|C{1.8cm}|}
        \hline
        $U$, В & $I_{L_{\text{кр.}}}$, мкА & $B_\text{кр.}$, мкТл & $e/m$, Кл/кг  \\
        \hline
        & & & \\
        \hline
        & & & \\
        \hline
        & & & \\
        \hline
    \end{tabular}

    \smallvspace

    \textit{Таблица \placeholder{M}.} \placeholder{Какие-то интересные расчёты}

\end{center}

    \clearpage

    \begin{center}
        \LARGE
        \textbf{Приложение 3. Графики}
    \end{center}

    \mediumvspace

    \hypertarget{diagram1}{}

\begin{center}
    \includegraphics[width=15cm]{images/graphic1}

    \smallvspace

    \textit{График 1.} Зависимость $I_a = I_a(I_L)$
\end{center}

\hypertarget{diagram2}{}

\begin{center}
    \includegraphics[width=15cm]{images/graphic2}

    \smallvspace

    \textit{График 2.} Зависимость $I_a/I_L$ от $I_L$
\end{center}

\hypertarget{diagram3}{}

\begin{center}
    \includegraphics[width=15cm]{images/graphic3}

    \smallvspace

    \textit{График 3.} Зависимость $B^2_\text{кр}$ от $U$
\end{center}



\end{document}