\begin{itemize}
    \item Заметим, что левая часть графика имеет теоретическую линейную зависимость. В силу 
    явлений, имеющих сложную природу описания, правую часть вычислений будем аппроксимировать 
    полиномом 3-ей степени.

    \item Найдем линейную зависимость для первых 8 полученных значения с помощью метода наименьших квадратов. 
    Пример для $U = 9$ В:

    $\overline{I_{L1-8}} = 82.5, \qquad \overline{I_{a1-8}} = 0.218$

    $\displaystyle b = \frac{\sum_{i = 1}^8 (I_{Li} - \overline{I_{L1-8}})(I_{ai} - \overline{I_{a1-8}})}{\sum_{i = 1}^8 (I_{Li} - \overline{I_{L1-8}})^2} = -1.19 \cdot 10^{-6}$

    $a = \overline{I_{a1-8}} - b \cdot \overline{I_{L1-8}} = 0.218$

    Получаем данные линейные аппроксимации:

    \begin{center}
        \begin{tabular}{|c|c|}
            \hline
            $U$, В & Линейная аппроксимация ($I_L$ и $I_a$ в мА) \\
            \hline
            $9$ & $I_a \approx -1.19 \cdot 10^{-6} \cdot I_L + 0.218$ \\
            \hline
            $10.5$ & $I_a \approx -5.89 \cdot 10^{-6} \cdot I_L + 0.264$ \\
            \hline
            $13.5$ & $I_a \approx -1.66 \cdot 10^{-5} \cdot I_L + 0.357$ \\
            \hline
        \end{tabular}

        \smallvspace

        \textit{Таблица 2.} Аппроксимации $I_a = I_a(I_L)$ при $B < B_\text{кр}$

    \end{center}

    \item Найдем полиномы 3-ей степени для аппроксимации зависимости для $B > B_\text{кр}$. 
    Решая данную систему уравнений для $a, b, c, d$

    \[\begin{cases} \displaystyle 
        a \sum_{i = 9}^{20} I_{Li}^3 + b \sum_{i = 9}^{20} I_{Li}^2 + c \sum_{i = 9}^{20} I_{Li} + d n = \sum_{i = 9}^{20} I_{Li}   \\
        a \sum_{i = 9}^{20} I_{Li}^4 + b \sum_{i = 9}^{20} I_{Li}^3 + c \sum_{i = 9}^{20} I_{Li}^2 + d \sum_{i = 9}^{20} I_{Li}^1 = \sum_{i = 9}^{20} I_{ai} I_{Li}   \\
        a \sum_{i = 9}^{20} I_{Li}^5 + b \sum_{i = 9}^{20} I_{Li}^4 + c \sum_{i = 9}^{20} I_{Li}^3 + d \sum_{i = 9}^{20} I_{Li}^2 = \sum_{i = 9}^{20} I_{ai}^2 I_{Li}   \\
        a \sum_{i = 9}^{20} I_{Li}^6 + b \sum_{i = 9}^{20} I_{Li}^5 + c \sum_{i = 9}^{20} I_{Li}^4 + d \sum_{i = 9}^{20} I_{Li}^3 = \sum_{i = 9}^{20} I_{ai}^3 I_{Li}  \\
    \end{cases}\]

    получаем такие полиномы 3-ей степени:
    
    \begin{center}
        \begin{tabular}{|c|c|}
            \hline
            $U$, В & Кубическая аппроксимация ($I_L$ и $I_a$ в мА) \\
            \hline
            $9$ & $I_a \approx -5.89 \cdot 10^{-9} \cdot I_L^3 + 8.91 \cdot 10^{-6} \cdot I_L^2 + 
                -4.51 \cdot 10^{-3} \cdot I_L + 0.802$ \\
            \hline
            $10.5$ & $I_a \approx -1.72 \cdot 10^{-8} \cdot I_L^3 + 2.14 \cdot 10^{-5} \cdot I_L^2 + 
                -9.16 \cdot 10^{-3} \cdot I_L + 1.397$ \\
            \hline
            $13.5$ & $I_a \approx -3.79 \cdot 10^{-8} \cdot I_L^3 + 4.54 \cdot 10^{-5} \cdot I_L^2 + 
                -1.84 \cdot 10^{-6} \cdot I_L + 2.638$ \\
            \hline
        \end{tabular}

        \smallvspace

        \textit{Таблица 3.} Аппроксимации $I_a = I_a(I_L)$ при $B > B_\text{кр}$

    \end{center}

    \item Пересечения данных функций дают такие точки: 
    
    \begin{itemize}
        \item для $U = 9$ В $\ I_{L_\text{кр}} = 194.31$ мА
        \item для $U = 10.5$ В $\ I_{L_\text{кр}} = 208.09$ мА
        \item для $U = 13.5$ В $\ I_{L_\text{кр}} = 223.84$ мА
    \end{itemize}

    \item Для полученных значений критической силы тока вычисляем критическое значение магнитного поля
    и удельный заряд электрона. Пример для $U = 9$ В:

    $\displaystyle B_\text{кр} = \mu_0 I_{L_\text{кр}} N \frac{1}{\sqrt{l^2 + d^2}} = 7.094 \cdot 10^{-3}$ Тл

    $\displaystyle \frac{e}{m} = \frac{8 U}{B^2_\text{кр} r^2_a} = 1.589 \cdot 10^{11}$ Кл/кг

    Результаты прилагаются в \hyperlink{table2}{таблице 5} в Приложении 2

    \item При помощи метода наименьших квадратов найдем угловой коэффициент прямой, аппроксимирующей 
    зависимость $B^2_\text{кр}$ от $U$:

    $\overline{U} = 11, \qquad \overline{B^2_\text{кр}} = 5.829 \cdot 10^{-5}$

    $\displaystyle b = \frac{\sum_{i = 1}^3 (U_i - \overline{U})(B^2_{i\text{кр}} - \overline{B^2_\text{кр}})}{\sum_{i = 1}^3 (U_i - \overline{U})^2} = 3.568 \cdot 10^{-6}$

    $a = \overline{B^2_\text{кр}} - b \cdot \overline{U} = 1.903 \cdot 10^{-5}$

    \item Подставляя в формулу значение $\displaystyle \frac{\Delta B^2_\text{кр}}{\Delta U} = 3.568 \cdot 10^{-6}$ получаем $\displaystyle \frac{e}{m} = 2.490 \cdot 10^{11} $ Кл/кг

\end{itemize}
