\text{Сила Лоренца}
\[
    \vec{F}_L = \vec{F}_E + \vec{F}_M = -e\vec{E} -e [\vec{v}, \vec{B}]
\]

\text{Кинетическая энергия электрона в точке c потенциалом $\varphi$}
\[
    e\varphi = \frac{mv^2}{2} = \frac{m (v_r^2 + v_\varphi^2)}{2}
\]

\text{При критическом значении магнитной индукции $B_c$}
\[
    eU = \frac{m v_\varphi^2}{2}
\]

\text{Азимутальная скорость электрона}
\[
    v_\varphi = \frac{e B r}{2m}
\]

\text{Удельный заряда электрона}
\[
    \frac{e}{m} = \frac{8 U}{B_c^2 r_a^2}
\]

\text{Магнитная индукция внутри соленоида}
\[
    B_c = \mu_0 N \frac{I_c}{\sqrt{l^2 + d^2}}
\]

\text{Где:}
\begin{center}
    $\mu_0 = 4\pi\cdot10^{-7} \frac{\text{Гн}}{\text{м}} - \text{магнитаная постоянная}$ \\
    $N - \text{число витков соленоида}$ \\
    $l - \text{длина соленоида}$ \\
    $d - \text{диаметр соленоида}$
\end{center}

\text{Итоговая формула для расчета удельного заряда электрона}
\[
    \frac{e}{m} = \frac{8U (l^2 + d^2)}{(\mu_0 r_a N I_c)^2}
\]

\textbf{Исходные данные}
\begin{center}
    $r_a = 0.003$ м — радиус анода. \\
    $d = 37$ мм — диаметр соленоида. \\
    $l = 36$ мм — длина соленоида. \\
    $N = 1500$ — число витков. \\
\end{center}